% Based on autoCV by Jitin Nair
% https://www.overleaf.com/latex/templates/autocv/scfvqfpxncwb

% % fontspec allows you to use TTF/OTF fonts directly
% \usepackage{fontspec}
% \defaultfontfeatures{Ligatures=TeX}

% % modified for ShareLaTeX use
% \setmainfont[
% SmallCapsFont = Fontin-SmallCaps.otf,
% BoldFont = Fontin-Bold.otf,
% ItalicFont = Fontin-Italic.otf
% ]
% {Fontin.otf}
\documentclass[a4paper,10pt]{article}
\usepackage[english]{babel}
\usepackage[T1]{fontenc}
\usepackage{lmodern}
\usepackage{url}
\usepackage{parskip} 	

\RequirePackage{graphicx}
\usepackage[usenames,dvipsnames]{xcolor}
\usepackage[scale=0.9]{geometry}

\usepackage[style=authoryear,sorting=ynt, maxbibnames=2]{biblatex}
\usepackage{supertabular} %to prevent spillover of tabular into next pages
\usepackage{tabularx} %tabularx environment
\usepackage{enumitem}
\usepackage{dirtytalk} %quotation marks
\usepackage{ulem}
\usepackage{titlesec}				
\usepackage{multicol}
\usepackage{multirow}
\usepackage[official]{eurosym} %euro
\usepackage[unicode, draft=false]{hyperref}
\usepackage{textcomp} %registered symbols
\usepackage[super]{nth} %superscript nth
\usepackage{color} %colors
    \definecolor{custom}{rgb}{0,0.2,0.6}
    \definecolor{back}{rgb}{0.85, 0.89, 0.95}
    \definecolor{fore}{rgb}{0.56, 0.67, 0.86}
\usepackage{hyperref} %hyperlinks
   \hypersetup{colorlinks=true,allcolors=custom}
\usepackage{fontawesome5} %social icons

\newcolumntype{C}{>{\centering\arraybackslash}X} % centered version of 'X' col. type

\newlength{\fullcollw}
\setlength{\fullcollw}{0.47\textwidth}

\titleformat{\section}{\Large\scshape\raggedright}{}{0em}{}[\titlerule]
\titlespacing{\section}{0pt}{15pt}{15pt}

\setlength\bibitemsep{1em}

\begin{document}

\pagestyle{empty} 

% HEADER
\begin{tabularx}{\linewidth}{@{} C @{}}
\LARGE{\textsc{Romain Tilhac}} \\
Earth scientist $|$ Post-doctoral researcher (IACT/CSIC, Granada)\\[7.5pt]
\href{mailto:romain.tilhac@csic.es}{\raisebox{-0.05\height}\faEnvelope \ romain.tilhac@csic.es} \ $|$ \
\href{https://orcid.org/0000-0001-5132-6228}{\raisebox{-0.05\height}\faOrcid \ 0000-0001-5132-6228} \ $|$ \
\href{https://romaintilhac.github.io}{\raisebox{-0.05\height}\faGlobe \ romaintilhac.github.io} \ $|$ \ 
\href{https://github.com/romaintilhac}{\raisebox{-0.05\height}\faGithub\ romaintilhac}
\end{tabularx}

% RESEARCH INTERESTS
\section{Research interests}

    {My research focuses on the role of melt generation, migration, melt-rock interaction and kinetics in the evolution and dynamics of the Earth's mantle. My approach combines a wide range of analytical techniques and numerical models to develop a petrologically consistent approach to geochemistry. One of my main interest is the role of pyroxenite formation on the building arc lithosphere and the impact of its recycling in the genesis of oceanic basalts and during geochemical cycles.}

% EDUCATION
\section{Education}

    {\bf PhD in Petrology and Geochemistry}, Macquarie University
    \hfill {2013 - 2017}\\
    {Co-tutelle with Paul Sabatier University}
    \hfill \textit{Sydney, Australia}\\
    \say{\textit{Petrology and geochemistry of pyroxenites from the Cabo Ortegal Complex, Spain}}

    {\bf BSc \& MSc in Earth and Planetary Sciences}, Paul Sabatier University
    \hfill {2006 - 2011}\\
    \uline{Top of the year, with honors}
    \hfill \textit{Toulouse, France}
     
% EXPERIENCE
\section{Employment history}
    \textbf{Post-doctoral researcher \textit{JdC Fellow}}
    \hfill {Since 2020}\\
    Instituto Andaluz de Ciencias de la Tierra (IACT)/CSIC, with \textbf{C. Garrido}
    \hfill \textit{Granada, Spain}
     
    \textbf{Post-doctoral researcher \textit{JSPS Fellow}}
    \hfill {2020}\\
    Kanazawa University, with \textbf{T. Morishita}
    \hfill \textit{Kanazawa, Japan}
    
    \textbf{Research Associate}
    \hfill {2017 - 2019}\\
    ARC Centre of Excellence CCFS/GEMOC, with \textbf{S.Y. O'Reilly} and \textbf{W.L. Griffin}
    \hfill \textit{Sydney, Australia}\\
    \uline{Manager of the geochronology group TerraneChron\textsuperscript{\textregistered}}
    
% SKILLS
\section{Scientific skills}

    \textbf{Analytical techniques} 
    \begin{itemize}[itemsep=0pt,parsep=2pt]
        \item Igneous \& metamorphic petrography, mineralogy (microscopy, SEM, thermobarometry, micro-thermometry)
        \item Mineral separation (magnetic, heavy-liquid \& Selfrag disagregation, picking)
        \item Wet chemistry (acid digestion, column chromatography, solvent extraction, micro-distillation)
        \item Major- \& trace-element geochemistry (EMP, solution ICP-MS, LA-ICP-MS, mapping by LA-ICP-MSI)
        \item Isotope geochemistry \& geochronology (analysis of radiogenic isotopes Rb-Sr, Sm-Nd, Lu-Hf, Re-Os by TIMS (Triton) \& MC-ICP-MS (Nu Plasma, Neptune), U-Pb/Lu-Hf zircon dating by LA-[MC]-ICP-MS)
    \end{itemize}
    
    \textbf{Numerical techniques}
    \begin{itemize}[itemsep=0pt,parsep=2pt]
        \item Modelling of elemental \& isotopic fractionation associated with magmatic processes
        \item Development of diffusion, percolation-diffusion, open-system melting \& mixed-source melting models
        \item Thermo-mecanical (reactive transport) modelling, thermodynamic modelling (pMELTS, PerpleX, Melt-PX)
        \item Coding languages: Matlab, Python \& VBA (\textit{Geoch’Em} for Excel) + notions of Julia, Fortran \& HTML
        
    \end{itemize}
    
    \textbf{Field geology}
    \begin{itemize}[itemsep=0pt,parsep=2pt]
        \item Igneous and metamorphic petrology of mafic \& ultramafic terranes
        \item Field experiences: Pyrenees, Galicia, S. Spain, Italy, Czech Republic, California, Australia, Newfoundland
        \item Micro-tectonic analysis, sampling, mapping
    \end{itemize}

% SKILLS
\section{Languages}
    French (native language), English (fluent, IELTS = 8.0), Spanish (fluent)

% GRANTS & REWARDS
\section{Awards \& fundings}

    \textbf{Research grant OCEANS (Principal Investigator)}
    \hfill {Sept. 2022}\\
    \say{\textit{M\underline{o}delling ar\underline{c} r\underline{e}cycling in the oceanic m\underline{a}ntle using radioge\underline{n}ic i\underline{s}otope systems}}\\
    Spanish Ministry of Sciences, Innovation and Universities - 45 k\geneuronarrow{} (2 years)
    
    \textbf{Post-doctoral fellowship \textit{Juan de la Cierva (Incorporación)}}
    \hfill {Aug. 2021}\\
    Spanish Ministry of Sciences, Innovation and Universities (3 years)
    
    \textbf{Post-doctoral fellowship \textit{Juan de la Cierva (Formación)}}
    \hfill {Dec. 2019}\\
    Spanish Ministry of Sciences, Innovation and Universities (2 years)
    
    \textbf{Post-doctoral fellowship JSPS (Short-term)}
    \hfill {Oct. 2019}\\
    \textit{Japan Society for the Promotion of Science} (1 year)
    
    \textbf{PhD thesis ranked in the top 10\% thesis examined by the panel}
    \hfill {Sept. 2017}\\
    Macquarie University (Sydney)
    
    \textbf{Doctoral iMQRES scholarship}
    \hfill {Feb. 2012}\\
    International Macquarie Research Excellence Scholarship (3.5 years)

% SUPERVISION & TEACHING
\section{Teaching, supervision \& training}

\textbf{Teaching}, Macquarie University
    \begin{itemize}[label={},itemsep=0pt,parsep=0pt]
        \item Lectures (MSc, 8h): Modelling of trace-element fractionation during magmatic processes \hfill 2017 - 2019 
        \item Field tutoring (BSc, 50h + exam marking): Structural and metamorphic geology (Hill End, NSW) \hfill 2014 - 2015 
    \end{itemize}
    
\textbf{Supervision}, Macquarie University
    \begin{itemize}[label={},itemsep=0pt,parsep=0pt]
    \item PhD thesis (H. Henry)
        \say{\textit{Mantle pyroxenites: deformation and seismic properties}}
        \hfill {2015 - 2018}
    \item MSc thesis (M. Smith)
        \say{\textit{Dating the Donkerhuk granite, Damara Orogen, Namibia}}
        \hfill {2018}
    \end{itemize}
    
\textbf{Delivered training / Workshop organization}
    \begin{itemize}[label={},itemsep=0pt,parsep=0pt]
        \item \textit{\nth{7} Orogenic Lherzolite conference}: Field trip to the Cabo Ortegal Complex
        \hfill \textit{Oviedo}, {2024}
        \item CNRS \textit{Forsterite 2021} workshop: Modelling of crust-mantle elemental transfers
        \hfill \textit{Pyrenees}, {2021}
        \item \textit{Geoanalysis 2018} workshop: Application of LA-[MC]-ICP-MS to exploration needs
        \hfill \textit{Sydney}, {2018}
    \end{itemize}
    
% OTHER ACTIVITIES
\section{Academic responsabilities \& leadership}

    \textbf{Invited seminars}
    \begin{itemize}[label={},itemsep=0pt,parsep=0pt]
        \item Goethe University geosciences colloquium series
            \hfill \textit{Frankfurt}, {2023}
        \item International Symposium \textit{DEEP 2021}
            \hfill \textit{Nanjing}, {2021}
        \item University of Tokyo
            \hfill \textit{Tokyo}, {2019}
    \end{itemize}

    \textbf{Sessions convened at the \textit{Goldschmidt Conference}}
    \begin{itemize}[label={},itemsep=0pt,parsep=0pt]
    \item 2f: Insights on the formation, preservation \& transport of mantle compositional heterogeneities
        \hfill {2023}
    \item 2e: Mantle heterogeneity: origins \& contribution to magmatism \& implications for mantle dynamics
        \hfill {2021}
    \item 3a: Development \& recycling of chemical \& isotopic heterogeneities in the sub-arc mantle
        \hfill {2020}
    \end{itemize}
    
    \textbf{Peer-reviews for international scientific journals}
    \begin{itemize}[label={},itemsep=0pt,parsep=0pt]
    \item 24 reviews to date
    \item \textit{Geology, Journal of Petrology, Earth-Science Reviews, Chemical Geology, Scientific Reports, GSL Special Publications, Lithos, European Journal of Mineralogy, EMU Notes in Mineralogy, American Journal of Science, Frontiers, Ofioliti}
    \end{itemize}
% ADDITIONAL TRAINING
\section{Additional training}

    \begin{itemize}[label={},itemsep=0pt,parsep=0pt]
        \item LA-ICP-MS imaging applications in petrology \& volcanology (M. Petrelli, C. Stremtan, M. Šala) \hfill {2023}
        \item Oxygen fugacity: theory \& practices in geosciences (C.A. McCammon, H.St.C. O’Neill, D.J. Frost) \hfill {2014}
        \item Geochemical analysis \& techniques (N.J. Pearson) / Geophysics \& geodynamics frontiers (C.J. O’Neil) \hfill {2014}
    \end{itemize}
    
%    
%% LANGUAGES
%\section{Languages}
%
%French, English, Spanish

\vfill\center{\footnotesize Updated on \today}

\end{document}
